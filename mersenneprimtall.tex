\documentclass[a4paper,twoside]{article}
\usepackage{graphicx, fullpage, float, subfig, verbatim, chemfig,amsmath,amssymb,amsfonts,array,hyperref,refcount, commath,amsthm}
\theoremstyle{plain}
\newtheorem{thm}{Teorem}[section]
\newtheorem{col}[thm]{Korrolar}
\newtheorem{defi}[thm]{Definisjon}
\newtheorem{setn}[thm]{Setning}
\newtheorem{hypo}[thm]{Hypotese}
\newtheorem{lemma}[thm]{Lemma}
\usepackage[utf8]{inputenc} %For "spesielle" tegn som æ, ø, å og andre er det anbefalt å angi dette. Mac brukere kan vurdere applemac og ikke utf8
\usepackage[norsk]{babel} %Inkluder kun dersom du vil skrive rapporten på norsk. Dette gir riktig datoformat og sørger for andre lokaliseringsting.
\newcommand{\setZ}{\mathbb{Z}}
\newcommand{\setNp}{\mathbb{N}^+}
\newcommand{\setNn}{\mathbb{N}_0}
\newcommand{\setR}{\mathbb{R}}
\newcommand{\setC}{\mathbb{C}}
\newcommand{\floor}[1]{\left \lfloor #1 \right \rfloor}
\newcommand{\ceil}[1]{\left \lceil #1  \right \rceil}
\newcommand{\nequiv}{\not\equiv}
\newcommand{\sgnp}[1]{\text{sgn}^{+}(#1)}
\newcommand{\sgn}[1]{\text{sgn}(#1)}
\newcommand{\eqtag}{\refstepcounter{equation}\tag{\theequation}}
\newcounter{reset}
\newcommand{\texteqtag}[1]{\refstepcounter{equation}(\theequation)\setcounter{reset}{\value{section}}\setcounter{section}{\value{equation}}\label{#1}\setcounter{section}{\value{reset}}\relax}
\newcommand{\eq}{$$\Updownarrow$$}
\title{Mersennetall og deres primtallsfaktorer}
\author{Jakob Peder Pettersen}
%\author{Kai M\"{u}ller Beckwith \and Vegar Ottesen}
\date{\today}
\begin{document}
\maketitle
\begin{abstract}
Dette beviset tar for seg Mersennetallene og deres primtallsfaktorer. Spesielt handler dette teoremet om hvilke Mersennetall som er primtall.
\end{abstract}
\tableofcontents \newpage
\section{Definisjoner brukt i beviset}
\subsection{Mersennetallene}
\subsubsection{Grunnleggende definisjon}
For $n\geq 1$, er det $n$-te Mersennetallet ($M_n$) definert ved:
\[
M_n=2^{n}-1
\]
Tallet $n$ kalles heretter indekser til Mersennetallet.
\subsubsection{Mersenneprimtall}
Dersom $M_n$ er et primtall, kalles da $M_n$ et Mersenneprimtall. $n$ blir da kalt en primtallgenerende indeks.
\subsection{Største felles divisor}
Største felles divisor mellom to heltall $a$ og $b$, notert som $\gcd\left(a,b\right)$ er definert som det største naturlige tallet som deler både $a$ og $b$.
\subsection{Eulers $\phi$-funksjon}
La $n$ være et naturlig tall. Da er $\phi(n)$ definert som antall naturlige tall $k$ slik $\gcd(n,k)=1$
\subsection{Delelighet}
Dersom $a,b \in \setZ$, betyr $a \mid b$ at $a$ deler $b$, det vil si at det finnes et tall $k\in \setZ$ slik at $b=k\cdot a$. I motsatt fall sier vi at $a$ ikke deler $b$ og skriver $a\nmid b$.
\section{Teori brukt i beviset}
\subsection{Eulers teorem}
Dersom $\gcd(a,n)$, så gjelder:
\begin{align}
	a^{\phi(n)} \equiv 1 \mod n
\end{align}

\section{Felles faktorer mellom Mersennetall}
\begin{thm}
	Største felles devisor mellom to Mersennetall er selv et Mersennetall, nærmere bestemt det Mersennetallet som har indeksen som er største felles divisor til indeksene til de to opprinnelige  Mersennetallene, altså:\[\gcd\left(M_a,M_b\right)=M_{\gcd(a,b)} \]
\end{thm}
For å vise dette trenger vi følgende lemma:
\begin{lemma}
	For alle $a,b \in \setNp$ så er:
\[ M_a\mid M_{a\cdot b} \]
\end{lemma}
Bevis:
Opplagt gjelder:
\begin{align}
M_a = 2^a-1\equiv 0 \pmod{M_a}
\end{align}
Dette kan omformes til:
\begin{align}
2^a \equiv 1  \pmod{M_a}
\end{align}
og igjen til:
\begin{align}
\left(2^a\right)^b\equiv 1^b\equiv 1 \pmod{M_a}
\end{align}
Dette er det samme som:
\begin{align}
M_{a\cdot b}=2^{a\cdot b}-1\equiv 0 \pmod{M_a}
\end{align}
og følgelig:
 \begin{align}
 M_a\mid M_{a\cdot b}
 \end{align}
som bekrefter lemmaet vårt. \bigskip


Videre til hovedresultatet går beviset som følger:
\begin{proof}
Vi vet at vi kan skrive:
\begin{align}
\gcd(a,b)=t\cdot a+s\cdot b
\end{align}
der $t,s \in \setZ$.
Imidlertid må nøyaktig ett av tallene $s$ og $t$ være positivt og det andre må være null eller negativt. Dersom det ene tallet er null, vil vi da imidlertid ha at 
\begin{align}
\gcd(a,b)=k\cdot \min(a,b)
\end{align}
, noe som gir:
\begin{align}
\gcd(a,b)=\min(a,b)
\end{align}
og følgelig
$\min(a,b)\mid \max(a,b)$. I dette spesialtilfellet sørger lemmaet for at $M_{\min(a,b)}\mid M_{\max(a,b)}$ og følgelig at 
\[\gcd\left(M_a,M_b\right)=M_{\gcd(a,b)} \]

Dersom ingen av tallene $s$ og $t$ er null, la da $d$ være en felles divisor av $M_a$ og $M_b$. Da vet vi:
$d\mid M_{|t|\cdot a}$ og $d\mid M_{|s|\cdot b}$ (merk absoluttverditegnene). Av dette har vi at $d\mid \left| M_{|t|\cdot a}-M_{|s|\cdot b}\right|$. Følgelig blir:
\begin{align}
\left| M_{|t|\cdot a}-M_{|s|\cdot b}\right|= \left| \left(2^{|t|\cdot a}-1\right)- \left(2^{|s|\cdot b}-1\right)\right|=\left| 2^{|t|\cdot a} -2^{|s|\cdot b}\right| =\\ \left| 2^{\min\left(|t|\cdot a,|s|\cdot b\right)}\cdot\left(2^{\max\left(|t|\cdot a,|s|\cdot b\right)-\min\left(|t|\cdot a,|s|\cdot b\right)}-1\right) \right|= \left| 2^{\min\left(|t|\cdot a,|s|\cdot b\right)}\cdot\left(2^{\left| |t|\cdot a-|s|\cdot b\right|}-1\right) \right|=\\
\left| 2^{\min\left(|t|\cdot a,|s|\cdot b\right)}\cdot\left(2^{t\cdot a+s\cdot b}-1\right) \right|=2^{\min\left(|t|\cdot a,|s|\cdot b\right)}\cdot M_{t\cdot a+s\cdot b}= \\ 2^{\min\left(|t|\cdot a,|s|\cdot b\right)}\cdot M_{\gcd(a,b)}
\end{align}
Altså har vi da at $d \mid 2^{\min\left(|t|\cdot a,|s|\cdot b\right)}\cdot M_{\gcd(a,b)}$, men Mersennetallene er alltid oddetall, så  $\gcd\left(d, 2^{\min\left(|t|\cdot a,|s|\cdot b\right)}\right)=1$, så vi får:
$d \mid M_{\gcd(a,b)}$. På den annen side har vi av lemmaet at $M_{\gcd(a,b)} \mid M_a$ og $M_{\gcd(a,b)} \mid M_b$. Altså må vi da ha at
\begin{align}
\gcd\left(M_a,M_b\right)=M_{\gcd(a,b)}
\end{align}
\end{proof}

\end{document}