\documentclass[a4paper,twoside]{article}
\usepackage{graphicx, fullpage, float, subfig, verbatim, chemfig,amsmath,amssymb,amsfonts,array,hyperref,refcount, commath,amsthm}
\theoremstyle{plain}
\newtheorem{thm}{Teorem}[section]
\newtheorem{col}[thm]{Korrolar}
\newtheorem{defi}[thm]{Definisjon}
\newtheorem{setn}[thm]{Setning}
\newtheorem{hypo}[thm]{Hypotese}
\newtheorem{lemma}[thm]{Lemma}
\usepackage[utf8]{inputenc} %For "spesielle" tegn som æ, ø, å og andre er det anbefalt å angi dette. Mac brukere kan vurdere applemac og ikke utf8
\usepackage[norsk]{babel} %Inkluder kun dersom du vil skrive rapporten på norsk. Dette gir riktig datoformat og sørger for andre lokaliseringsting.
\newcommand{\setZ}{\mathbb{Z}}
\newcommand{\setNp}{\mathbb{N}^+}
\newcommand{\setNn}{\mathbb{N}_0}
\newcommand{\setR}{\mathbb{R}}
\newcommand{\setC}{\mathbb{C}}
\newcommand{\floor}[1]{\left \lfloor #1 \right \rfloor}
\newcommand{\ceil}[1]{\left \lceil #1  \right \rceil}
\newcommand{\nequiv}{\not\equiv}
\newcommand{\sgnp}[1]{\text{sgn}^{+}(#1)}
\newcommand{\sgn}[1]{\text{sgn}(#1)}
\newcommand{\eqtag}{\refstepcounter{equation}\tag{\theequation}}
\newcounter{reset}
\newcommand{\texteqtag}[1]{\refstepcounter{equation}(\theequation)\setcounter{reset}{\value{section}}\setcounter{section}{\value{equation}}\label{#1}\setcounter{section}{\value{reset}}\relax}
\newcommand{\eq}{$$\Updownarrow$$}
\title{Mersennetall og deres primtallsfaktorer}
\author{Jakob Peder Pettersen}
%\author{Kai M\"{u}ller Beckwith \and Vegar Ottesen}
\date{\today}
\begin{document}
\maketitle
\begin{abstract}
Dette beviset tar for seg Mersennetallene og deres primtallsfaktorer. Spesielt handler dette teoremet om hvilke Mersennetall som er primtall.
\end{abstract}
\tableofcontents \newpage
\section{Definisjoner brukt i beviset}
\subsection{Mersennetallene}
\subsubsection{Grunnleggende definisjon}
For $n\geq 1$, er det $n$-te Mersennetallet ($M_n$) definert ved:
\[
M_n=2^{n}-1
\]
Tallet $n$ kalles heretter indekser til Mersennetallet.
\subsubsection{Mersenneprimtall}
Dersom $M_n$ er et primtall, kalles da $M_n$ et Mersenneprimtall. $n$ blir da kalt en primtallgenerende indeks.
\subsection{Største felles divisor}
Største felles divisor mellom to heltall $a$ og $b$, notert som $\gcd\left(a,b\right)$ er definert som det største naturlige tallet som deler både $a$ og $b$.
\subsection{Eulers $\phi$-funksjon}
La $n$ være et naturlig tall. Da er $\phi(n)$ definert som antall naturlige tall $k$ slik $\gcd(n,k)=1$
\section{Teori brukt i beviset}
\subsection{Eulers teorem}
Dersom $\gcd(a,n)$, så gjelder:
\begin{align}
	a^{\phi(n)} \equiv 1 \mod n
\end{align}
\section{Felles faktorer mellom Mersennetall}
\begin{thm}
	Største felles devisor mellom to Mersennetall er selv et Mersennetall, nærmere bestemt det Mersennetallet som har indeksen som er største felles divisor til indeksene til de to opprinnelige  Mersennetallene, altså:\[\gcd\left(M_a,M_b\right)=M_{\gcd(a,b)} \]
\end{thm}


\end{document}